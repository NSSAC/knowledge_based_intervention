%%%%%%%%%%%%%%%%%%%%%%%%%%%%%%%%%%%%%%%%%%%%%%%%%%%%%%%%%%%%%%%%%%%%%%%  
%%% SIR Model with Knowledge-Based Interventions.  
%%%%%%%%%%%%%%%%%%%%%%%%%%%%%%%%%%%%%%%%%%%%%%%%%%%%%%%%%%%%%%%%%%%%%%%%  
  
\documentclass[11pt]{article}
  
\setlength{\textheight}{9.0 in}
\setlength{\textwidth}{6.5 in}
\setlength{\oddsidemargin}{0 in}
\setlength{\evensidemargin}{0 in}
\setlength{\topmargin}{-0.5 in}

\setlength{\parskip}{4pt}

\usepackage{times}  
\usepackage{amsfonts}  
\usepackage{latexsym}  
\usepackage{amsmath}  
\usepackage[latin1]{inputenc}  
\usepackage{varioref}  
  
\usepackage{fancybox}  
\usepackage{shadow}  
\usepackage{times}  
\usepackage{amsmath}  
\usepackage{amsfonts}  
\usepackage{latexsym}  
\usepackage{color}  
\usepackage{graphicx}  
%\usepackage{pstricks}  
  
\newtheorem{theorem}{Theorem}[section]  
\newtheorem{lemma}[theorem]{Lemma}  
\newtheorem{corollary}[theorem]{Corollary}  
\newtheorem{fact}[theorem]{Fact}  
\newtheorem{claim}[theorem]{Claim}  
\newtheorem{observation}[theorem]{Observation}  
\newtheorem{definition}[theorem]{Definition}  
\newtheorem{proposition}[theorem]{Proposition}  
\newtheorem{example}[theorem]{Example}  

\newcommand{\cnp}{\textbf{NP}}
\newcommand{\true}{\texttt{True}}
\newcommand{\false}{\texttt{False}}
  
\newcommand{\calb}{\mbox{${\cal B}$}}  
\newcommand{\calc}{\mbox{${\cal C}$}}  
\newcommand{\calcp}{\mbox{${\cal C}'$}}  

\newcommand{\calco}{\mbox{$\mathcal{C}_1$}}
\newcommand{\calct}{\mbox{$\mathcal{C}_2$}}
\newcommand{\calcop}{\mbox{$\mathcal{C}_1'$}}
\newcommand{\calcodp}{\mbox{$\mathcal{C}_1''$}}

\newcommand{\calcdp}{\mbox{${\cal C}''$}}  
\newcommand{\calf}{\mbox{${\cal F}$}}  
\newcommand{\cali}{\mbox{${\cal I}$}}  
\newcommand{\calp}{\mbox{${\cal P}$}}  
\newcommand{\cals}{\mbox{${\cal S}$}}  
\newcommand{\calt}{\mbox{${\cal T}$}}  
  
\newcommand{\reach}{{\textsc{Reachability}}}
\newcommand{\treach}{{$t$-\textsc{Reachability}}}

 % box for end of proof  
\newcommand{\QED}{\hfill\rule{2mm}{2mm}\medskip}   

\newcommand{\irange}{\mbox{$1 \leq i \leq n$}}
\newcommand{\jrange}{\mbox{$1 \leq j \leq m$}}

\newcommand{\dunder}[1]{\underline{\underline{#1}}}

 
%%% New commands for the three state values.
\newcommand{\sstate}{\mbox{$\mathbb{S}$}}
\newcommand{\istate}{\mbox{$\mathbb{I}$}}
\newcommand{\rstate}{\mbox{$\mathbb{R}$}}

%% New command for the set with the above three values.
\newcommand{\bset}{\mbox{\{\sstate, \istate, \rstate\}}} 
 
 
  
\title{\textbf{SIR Model With Knowledge-Based Interventions}}

%\vspace*{2ex}}  

\author{  
\vspace*{1ex}  
}  
  
  
\begin{document} 
 
\maketitle  
 
\baselineskip=1.4\normalbaselineskip

\section{Introduction}
  
  We consider the SIR model when there are interventions chosen based on 
  knowledge about the current configuration of the system.
  In particular, we consider the case where the decision to choose an intervention is based on 
  global knowledge of the overall configuration.
  We consider a system with only one available intervention, 
  but this assumption can be generalized to multiple available interventions.
  
  At each time step, we assume that the intervention is chosen whenever the total number of nodes 
  in the infected state at that time step is at above a threshold value.
  The value of the threshold could be specified as either a constant,
  or as a fraction of the total number of nodes.
  Because we consider the computational complexity of analyzing a given system for properties of interest,
  we find it more appropriate to specify the threshold as a fraction.
  
  Note that we assume the system adopts to the disease spread by choosing to utilize the intervention based on 
  global knowledge about the current configuration of the system.
  The decision to utilize the intervention could be made individually by each node,
  or centrally by appropriate authorities.
  
  A possible variation in the model is that if the threshold is reached, the intervention has to be maintained for 
  at least a required minimum number of time steps.
  
  We assume that the intervention reduces the transmission probability of each edge by a given fraction.
  An additional possibility is that a specified set of edges is deleted when the intervention occurs.
  Another possibility is that each edge has two probabilities:
  a base probability for when the intervention is not taking place,
  and a reduced probability for when the intervention is indeed taking place.
  This can model the situation when people who are more vulnerable 
  due to such factors as age and pre-existing conditions
  make more drastic modifications to their behavior when 
  there are a large number of infected individuals in the general population.
  
\section{Formal Model}
\label{sec:formal_model}

We represent an SIR
system by the underlying graph $G(V,E)$, with
a given {\bf base} transmission probability $p_e$ for each
edge $e \in E$. 
We let $N_V$ denote the number of nodes in $G$.
 
 The intervention is characterized by two fractions:
a {\bf transmission reduction factor} $\alpha$, 
and a {\bf threshold factor} $\beta$.
We require that $0 < \alpha <1$, and $0 < \beta <1$.

If, at a given time $t$, the number of nodes that are infected {\bf at} time $t$,
i.e. the number of nodes that are in state $\istate$ at time $t$,
 is at least $\beta N_V$,
then the transmission probability of each edge $e$ is reduced to $\alpha \, p_e$.

We consider the complexity of analyzing a given system for some property of interest.
We can consider special cases, such as when all the transmission probabilities are equal,
or when the underlying graph $G$ has some some special graph property.
We can also consider the special cases that arise when certain parameters are fixed.
Such parameters can include $\alpha$, $\beta$, and the base transmission probability.


\end{document}
