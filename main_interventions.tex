%%%%%%%%%%%%%%%%%%%%%%%%%%%%%%%%%%%%%%%%%%%%%%%%%%%%%%%%%%%%%%%%%%%%%%%  
%%% SIR Model with Knowledge-Based Interventions.  
%%%%%%%%%%%%%%%%%%%%%%%%%%%%%%%%%%%%%%%%%%%%%%%%%%%%%%%%%%%%%%%%%%%%%%%%  
  
\documentclass[11pt]{article}
  
\setlength{\textheight}{9.0 in}
\setlength{\textwidth}{6.5 in}
\setlength{\oddsidemargin}{0 in}
\setlength{\evensidemargin}{0 in}
\setlength{\topmargin}{-0.5 in}

\setlength{\parskip}{4pt}

\usepackage{times}  
\usepackage{amsfonts}  
\usepackage{latexsym}  
\usepackage{amsmath}  
\usepackage[latin1]{inputenc}  
\usepackage{varioref}  
  
\usepackage{fancybox}  
\usepackage{shadow}  
\usepackage{times}  
\usepackage{amsmath}  
\usepackage{amsfonts}  
\usepackage{latexsym}  
\usepackage{color}  
\usepackage{graphicx}  
%\usepackage{pstricks}  
  
\newtheorem{theorem}{Theorem}[section]  
\newtheorem{lemma}[theorem]{Lemma}  
\newtheorem{corollary}[theorem]{Corollary}  
\newtheorem{fact}[theorem]{Fact}  
\newtheorem{claim}[theorem]{Claim}  
\newtheorem{observation}[theorem]{Observation}  
\newtheorem{definition}[theorem]{Definition}  
\newtheorem{proposition}[theorem]{Proposition}  
\newtheorem{example}[theorem]{Example}  

\newcommand{\cnp}{\textbf{NP}}
\newcommand{\true}{\texttt{True}}
\newcommand{\false}{\texttt{False}}
  
\newcommand{\calb}{\mbox{${\cal B}$}}  
\newcommand{\calc}{\mbox{${\cal C}$}}  
\newcommand{\calcp}{\mbox{${\cal C}'$}}  

\newcommand{\calco}{\mbox{$\mathcal{C}_1$}}
\newcommand{\calct}{\mbox{$\mathcal{C}_2$}}
\newcommand{\calcop}{\mbox{$\mathcal{C}_1'$}}
\newcommand{\calcodp}{\mbox{$\mathcal{C}_1''$}}

\newcommand{\calcdp}{\mbox{${\cal C}''$}}  
\newcommand{\calf}{\mbox{${\cal F}$}}  
\newcommand{\cali}{\mbox{${\cal I}$}}  
\newcommand{\calp}{\mbox{${\cal P}$}}  
\newcommand{\cals}{\mbox{${\cal S}$}}  
\newcommand{\calt}{\mbox{${\cal T}$}}  
  
\newcommand{\reach}{{\textsc{Reachability}}}
\newcommand{\treach}{{$t$-\textsc{Reachability}}}

 % box for end of proof  
\newcommand{\QED}{\hfill\rule{2mm}{2mm}\medskip}   

\newcommand{\irange}{\mbox{$1 \leq i \leq n$}}
\newcommand{\jrange}{\mbox{$1 \leq j \leq m$}}

\newcommand{\dunder}[1]{\underline{\underline{#1}}}

 
%%% New commands for the three state values.
\newcommand{\sstate}{\mbox{$\mathbb{S}$}}
\newcommand{\istate}{\mbox{$\mathbb{I}$}}
\newcommand{\rstate}{\mbox{$\mathbb{R}$}}

%% New command for the set with the above three values.
\newcommand{\bset}{\mbox{\{\sstate, \istate, \rstate\}}} 
 
%%% General forms. -- New version (July 25, 2017)
\newcommand{\tNewInfs}{\mbox{\textsc{Pr-Num-Inf-at}$\,(t, q, S)$}}
\newcommand{\tNewInfv}{\mbox{\textsc{Pr-Num-Inf-at}$\,(t, q, V)$}}
\newcommand{\tTotInfs}{\mbox{\textsc{Pr-Num-Inf-by}$\,(t, q, S)$}}
\newcommand{\tTotInfv}{\mbox{\textsc{Pr-Num-Inf-by}$\,(t, q, V)$}}
\newcommand{\tPeak}{\mbox{\textsc{Pr-Peak-Inf-at}$\,(t)$}}
\newcommand{\tVuls}{\mbox{\textsc{Pr-Inf-at}\mbox{$\,(t, S)$}}}
\newcommand{\tVulv}{\mbox{\textsc{Pr-Inf-at}\mbox{$\,(t, V)$}}}
\newcommand{\tTotVuls}{\mbox{\textsc{Pr-Inf-by}\mbox{$\,(t, S)$}}}
\newcommand{\tTotVulv}{\mbox{\textsc{Pr-Inf-by}\mbox{$\,(t, V)$}}}

%%% Versions with t = 2. (New: July 25, 2017)
\newcommand{\TwoNewInfs}{\mbox{\textsc{Pr-Num-Inf-at}$\,(2, q, S)$}}
\newcommand{\TwoNewInfv}{\mbox{\textsc{Pr-Num-Inf-at}$\,(2, q, V)$}}
\newcommand{\TwoTotInfs}{\mbox{\textsc{Pr-Num-Inf-By}$\,(2, q, S)$}}
\newcommand{\TwoTotInfv}{\mbox{\textsc{Pr-Num-Inf-By}$\,(2, q, V)$}}
\newcommand{\TwoPeak}{\mbox{\textsc{Pr-Peak-Inf-at}$\,(2)$}}
\newcommand{\TwoVuls}{\mbox{\textsc{Pr-Inf-at}\mbox{$\,(2, S)$}}}
\newcommand{\TwoVulv}{\mbox{\textsc{Pr-Inf-at}\mbox{$\,(2, V)$}}}
\newcommand{\TwoTotVuls}{\mbox{\textsc{Pr-Inf-by}\mbox{$\,(2, S)$}}}
\newcommand{\TwoTotVulv}{\mbox{\textsc{Pr-Inf-by}\mbox{$\,(2, V)$}}}

%%% Versions with t = 1. (New: July 25, 2017)
\newcommand{\OneNewInfs}{\mbox{\textsc{Pr-Num-Inf-at}$\,(1, q, S)$}}
\newcommand{\OneNewInfv}{\mbox{\textsc{Pr-Num-Inf-at}$\,(1, q, V)$}}
\newcommand{\OneTotInfs}{\mbox{\textsc{Pr-Num-Inf-by}$\,(1, q, S)$}}
\newcommand{\OneTotInfv}{\mbox{\textsc{Pr-Num-Inf-by}$\,(1, q, V)$}}
\newcommand{\OnePeak}{\mbox{\textsc{Pr-Peak-Inf-at}$\,(1)$}}
\newcommand{\OneVuls}{\mbox{\textsc{Pr-Inf-at}\mbox{$\,(1, S)$}}}
\newcommand{\OneVulv}{\mbox{\textsc{Pr-Inf-at}\mbox{$\,(1, V)$}}}
\newcommand{\OneTotVuls}{\mbox{\textsc{Pr-Inf-by}\mbox{$\,(1, S)$}}}
\newcommand{\OneTotVulv}{\mbox{\textsc{Pr-Inf-by}\mbox{$\,(1, V)$}}}

%%% Version with t = 3. (New: July 25, 2017)
\newcommand{\ThrNewInfs}{\mbox{\textsc{Pr-Num-Inf-at}$\,(3, q, S)$}}
\newcommand{\ThrNewInfv}{\mbox{\textsc{Pr-Num-Inf-at}$\,(3, q, V)$}}
\newcommand{\ThrTotInfs}{\mbox{\textsc{Pr-Num-Inf-by}$\,(3, q, S)$}}
\newcommand{\ThrTotInfv}{\mbox{\textsc{Pr-Num-Inf-by}$\,(3, q, V)$}}

%%%  Problems Involving an Intervention
\newcommand{\Int}{\textsc{Pr-Intervention}}
\newcommand{\tActivatedAt}{\mbox{\textsc{Pr-Activated-at}$\,t$}}
\newcommand{\tActivatedBy}{\mbox{\textsc{Pr-Activated-by}$\,t$}}
\newcommand{\NumActivations}{\mbox{\textsc{Pr-Num-Activations}$\,q$}}



 
  
\title{\textbf{SIR Model With Knowledge-Based Interventions}}

%\vspace*{2ex}}  

\author{  
\vspace*{1ex}  
}  
  
  
\begin{document} 
 
\maketitle  
 
\baselineskip=1.4\normalbaselineskip

\section{Introduction}
\label{sec:introduction}
  
  We consider the SIR model when there are interventions chosen based on 
  knowledge about the current configuration of the system.
  In particular, we consider the case where the decision to choose an intervention is based on 
  global knowledge of the overall configuration.
  We consider a system with only one available intervention, 
  but this assumption can be generalized to multiple available interventions.
  
  At each time step, we assume that the intervention is chosen whenever the total number of nodes 
  in the infected state at that time step is at above a threshold value.
  The value of the threshold could be specified as either a constant,
  or as a fraction of the total number of nodes.
  Because we consider the computational complexity of analyzing a given system for properties of interest,
  we consider it more appropriate to specify the threshold as a fraction.
  
  Note that we assume the system adopts to the disease spread by choosing to utilize the intervention based on 
  global knowledge about the current configuration of the system.
  The decision to utilize the intervention could be made individually by each node,
  or centrally by appropriate authorities.
  
  A possible variation in the model is that if the threshold is reached, the intervention has to be maintained for 
  at least a required minimum number of time steps.
  
  We assume that the intervention reduces the transmission probability of each edge by a given fraction.
  An additional possibility is that a specified set of edges is deleted when the intervention occurs.
  Another possibility is that each edge has two probabilities:
  a base probability for when the intervention is not taking place,
  and a reduced probability for when the intervention is indeed taking place.
  This can model the situation when people who are more vulnerable 
  due to such factors as age and pre-existing conditions
  make more drastic modifications to their behavior when 
  there are a large number of infected individuals in the general population.
  
\section{Formal Model}
\label{sec:formal_model}

We represent an SIR
system by the underlying graph $G(V,E)$, with
a given {\bf base} transmission probability $p_e$ for each edge $e \in E$. 
%We let $N_V$ denote the number of nodes in $G$.
 
 The intervention is characterized by two fractions:
a {\bf transmission reduction factor} $\alpha$, 
and a {\bf threshold factor} $\beta$.
We require that $0 < \alpha <1$, and $0 < \beta <1$.

If, at a given time $t$, the number of nodes that are infected {\bf at} time $t$,
i.e. the number of nodes that are in state $\istate$ at time $t$,
 is at least $\beta |V|$,
then at time $t+1$, the transmission probability of each edge $e$ 
is the reduced value $\alpha \, p_e$.
In this case, 
we say that the intervention is {\bf activated} at time $t$ and {\bf utilized} at time $t+1$.

If, at a given time $t$, the number of nodes that are infected {\bf at} time $t$
is less than $\beta |V|$,
then at time $t+1$, the transmission probability of each edge $e$ is  $p_e$.
We assume that at time $t=0$, the intervention is not utilized, 
so all edges use their base transmission probability.
Subsequently, for each $t \geq 1$,
the intervention is utilized at time $t$ iff it is activated at time $t-1$.

We refer to a SIR system with a knowledge-based intervention available as a {\bf SIR-KBE} system.
A given SIR-KBE system is specified by underlying graph $G(V,E)$, 
value $p_e$ for each edge $e \in E$,
transmission reduction factor $\alpha$, and threshold factor $\beta$.



We consider the complexity of analyzing a given system for various properties of interest.
We can consider special cases, such as when all the transmission probabilities are equal,
or when the underlying graph $G$ has some some special graph property.
We can also consider the special cases that arise when certain parameters are fixed.
Such parameters can include $\alpha$, $\beta$, and the base transmission probability.

\section{Computational Problems for SIR-KBE Systems}
\label{sec:computational_problems}

All the computational problems for SIR systems 
are also well-defined and meaningful for SIR-KBE systems.
Moreover, there are additional problems pertaining to whether whether, when, and how often,
the intervention is utilized. Some possible problems of interest are the following.

\bigskip

\noindent
\textbf{Probability of an Intervention}~ (\Int)

\medskip
\noindent
\underline{Instance:}~ An SIR-KBE system $G(V,E)$ with a base transmission probability for each edge,
transmission reduction factor $\alpha$, threshold factor $\beta$;
along with initial configuration \cali.

\smallskip
\noindent
\underline{Requirement:}~ The probability of the following
event:~ the intervention gets activated at least once.

\bigskip

\noindent
\textbf{Number of Activations}~ ({\NumActivations})

\medskip
\noindent
\underline{Instance:}~ An SIR-KBE system $G(V,E)$ with a base transmission probability for each edge,
transmission reduction factor $\alpha$, threshold factor $\beta$;
along with initial configuration \cali; integer $q \geq 1$.

\smallskip
\noindent
\underline{Requirement:}~ The probability of the following
event:~ the intervention gets activated at least $q$ times.

\bigskip

\noindent
\textbf{Probability of an Intervention Activated At a Given Time}~ (\tActivatedAt)

\medskip
\noindent
\underline{Instance:}~ An SIR-KBE system $G(V,E)$ with a base transmission probability for each edge,
transmission reduction factor $\alpha$, threshold factor $\beta$;
along with initial configuration \cali; integer $t \geq 0$.

\smallskip
\noindent
\underline{Requirement:}~ The probability of the following
event:~ the intervention gets activated at time $t$.

\bigskip

\noindent
\textbf{Probability of an Intervention Activated By a Given Time}~ (\tActivatedBy)

\medskip
\noindent
\underline{Instance:}~ An SIR-KBE system $G(V,E)$ with a base transmission probability for each edge,
transmission reduction factor $\alpha$, threshold factor $\beta$;
along with initial configuration \cali; integer $t \geq 0$.

\smallskip
\noindent
\underline{Requirement:}~ The probability of the following
event:~ the intervention gets activated by time $t$.





\bigskip

\section{Computational Intractability Results for General Networks}
\label{sec:general_results}

The short-term forecasting reductions from our original paper 
are generally applicable to SIR-KBE systems,
since the reductions can construct problem instances where the intervention is never activated.
If the threshold factor $\beta$ is part of the constructed problem instance,
then it can be set so high that the intervention is never activated.

If we want to consider instances where $\beta$ is fixed,
then for problems involving the total number of infections and vulnerability,
the reduction can be modified by adding a sufficiently large number of nodes 
whose distance from the initially infected node is greater than the time $t$
in the short-term forecasting problem.
For the peak problem, 
the base transmission probability for the edges incident on these extra nodes can be made so small, 
that these nodes are very unlikely to be infected, 
and so these extra nodes do not affect when the peak occurs. 

\end{document}
